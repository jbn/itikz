% -*- coding: utf-8 ; -*-
\documentclass[notitlepage]{article}
\usepackage[french]{babel}
\pagestyle{empty}
\usepackage{xltxtra}
\usepackage{nicematrix,tikz}



\begin{document}

La commande \verb|\SubMatrix| est incluse dans l'extension \verb|nicematrix|. Il n'y plus besoin de demander la
création des nœuds moyens.

Pour les options, il n'y a plus de commande \verb|\SubMatrixOptions|. À la place, il y a des clés préfixées par \verb|sub-matrix|.
Ces clés sont \verb|extra-height|, \verb|left-xshift|, \verb|right-xshift| et \verb|xshift| (qui règle des deux
côtés à la fois).



\vspace{2mm}
Je n'ai pas de solution facile pour le problème du bloc vertical du 56.

Si on tient absolument à utiliser le bloc du 56 (au lieu d'utiliser un simple \verb|\boxed|), il faut renoncer à
utiliser le \verb|\noalign{\vskip2mm}| (qui de toutes manières est un réglage de bas niveau de TeX qui était une sorte d'astuce).

Je reviens donc à l'argument optionnel de la commande \verb|\\| (ici : \verb|\\[2mm]|).

On n'a pas de problème vertical avec le bloc qui contient 56, mais on ne peut pas utiliser le type de colonne
\verb|I| pour les filets verticaux discontinus.

Il reste une solution (pas très agréable) qui consiste à les tracer avec Tikz dans le \verb|\CodeAfter| (ce que j'ai fait ici).

De toutes manières, pour le moment, pour ce qui est des filets verticaux qui ne seraient tracés que dans les
sous-matrices, je n'ai que cette solution (néanmoins, ce serait effectivement une idée de programmer quelque chose
allant dans ce sens : je vais y réfléchir).

\medskip
% ================================================================================
$\begin{NiceArray}%
{r@{\qquad\ }*2r@{\qquad\ }*2r*1r@{\qquad\ }*1r*1r@{\qquad\ }*1r@{\qquad\ }rrl@{\qquad\;\;}}
 &  &  &  &  &  & 1 & 2 &  &  &  &  \\
 &  &  & 7 & 5 & 6 & 4 & 3 & \color{red}{3} &  &  &  \\
 &  &  & 4 & 8 & 1 & 2 & 5 & \color{red}{5} &  &  &  \\[2mm]
 & 4 & 7 & \Block[draw=black,fill=red!15]{}{56} & 76 & 31 & 422 & \boxed{\mathbf{495}} & 3741 &  &  &  \\
 & 1 & 3 & 19 & \Block[draw=black,fill=red!15]{1-1}{29} & 9 & \boxed{\mathbf{153}} & 170 & 1309 &  &  & 
\CodeAfter [ sub-matrix / xshift = 1mm ]
% ----------------------------------------- submatrix delimiters
    \SubMatrix({2-4}{3-6})
    \SubMatrix({1-7}{3-8})
    \SubMatrix({2-9}{3-9})
    \SubMatrix({4-2}{5-3})
    \SubMatrix({4-4}{5-6})
    \SubMatrix({4-7}{5-8})
    \SubMatrix({4-9}{5-9})
\begin{tikzpicture} % ----------------------------------- filets avec tikz
    \draw (row-2-|col-6) -- ([yshift=2mm]row-4-|col-6) ;
    \draw (row-4-|col-6) -- (row-6-|col-6) ;
    \draw (row-1-|col-8) -- ([yshift=2mm]row-4-|col-8) ;
    \draw (row-4-|col-8) -- (row-6-|col-8) ;
\end{tikzpicture}
\end{NiceArray}$
\vspace{2cm}

\newcolumntype{I}{!{\OnlyMainNiceMatrix{\vrule}}}
\end{document}
